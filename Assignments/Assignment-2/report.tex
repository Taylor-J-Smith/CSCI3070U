\documentclass[titlepage]{article}

\usepackage{titlesec}
\usepackage[margin=1in]{geometry}
\usepackage[pdftex]{graphicx}
\usepackage{changepage}
\usepackage{amsmath}
\usepackage{amssymb}

\titleformat*{\section}{\large\bfseries}
\titleformat*{\subsection}{\normalsize\bfseries}
\titleformat*{\subsubsection}{\normalsize\bfseries}
\titleformat*{\paragraph}{\normalsize\bfseries}
\titleformat*{\subparagraph}{\normalsize\bfseries}

\titlespacing*{\subsection} {2em}{3.25ex plus 1ex minus .2ex}{1.5ex plus .2ex}
\titlespacing*{\subsubsection} {3em}{3.25ex plus 1ex minus .2ex}{1.5ex plus .2ex}

\begin{document}
\title{CSCI3070U Assignment 2}
\date{Novemeber 22nd 2015}
\author{Taylor Smith \and 100372402}
\maketitle
\part*{Part A}

  \section{What is the eccentricity of a vertex in a graph. Illustrate with an example.}

    The eccentricity of a vertex is the greatest distance between it and any other vertex. It can be thought of as how far a node is from the farthest node in the graph.

    \begin{figure}[ht]
      \begin{minipage}[c]{0.45\linewidth}
      \centering
      \includegraphics[width=0.5\textwidth]{graph.png}
      \end{minipage}
      \hspace{0.5cm}
      \begin{minipage}[b]{0.45\linewidth}
      \centering
          \begin{tabular}{| c | c |}
            \hline
            Vertex & Eccentricity \\
            \hline
            1 & 3 \\
            \hline
            2 & 3 \\
            \hline
            3 & 2 \\
            \hline
            4 & 2 \\
            \hline
            5 & 2 \\
            \hline
            6 & 3 \\
            \hline
          \end{tabular}
      \end{minipage}
    \end{figure}

  \section{What is the radius of a graph. Illustrate with an example.}

    The radius of a graph is the minimum eccentricity of any vertex in the graph.\\
    In the example above the radius is 2.

  \section{What is the diameter of a graph. Illustrate with an example.}

    The diameter of the graph is the maximum eccentricity of any vertex in the graph.\\
    In the example above the diameter is 3.

  \section{Provide a detailed algorithmic solution to compute the \emph{three} properties. You may use functions such as BFS, DFS, and Dijkstra}

    \begin{minipage}[t]{0.3\linewidth}
      \subsection{Eccentricity}
        \begin{adjustwidth}{2em}{0pt}
          for v $\in$ V
          \begin{adjustwidth}{1em}{0pt}
            BFS(v) \\
            e[v] = max(d)
          \end{adjustwidth}
        \end{adjustwidth}
    \end{minipage}
\
    \begin{minipage}[t]{0.3\linewidth}
      \subsection{Radius}
        \begin{adjustwidth}{2em}{0pt}
          for v $\in$ V
          \begin{adjustwidth}{1em}{0pt}
            BFS(v) \\
            e[v] = max(d)
          \end{adjustwidth}
          r = min(e)
        \end{adjustwidth}
    \end{minipage}
\
    \begin{minipage}[t]{0.3\linewidth}
      \subsection{Diameter}
        \begin{adjustwidth}{2em}{0pt}
          for v $\in$ V
          \begin{adjustwidth}{1em}{0pt}
            BFS(v) \\
            e[v] = max(d)
          \end{adjustwidth}
          d = max(e)
        \end{adjustwidth}
    \end{minipage}

  \section{What are the computation complexities of the \emph{three} solutions?}

  \begin{minipage}[t]{0.33\linewidth}
    \subsection{Eccentricity}
      \begin{adjustwidth}{2em}{0pt}
        for v $\in$ V \hfill $\mathcal{O} \left( |V| \right)$
        \begin{adjustwidth}{1em}{0pt}
          BFS(v) \hfill $\mathcal{O} \left( |V| + |E| \right)$\\
          e[v] = max(d) \hfill $\mathcal{O} \left( |V| \right)$
        \end{adjustwidth}
        \ \\
        \ \\
        Overall: \\

        $ \mathcal{O} \left( |V| \times \left( \left( |V| + |E| \right) + |V| \right) \right)$
      \end{adjustwidth}
  \end{minipage}
\
  \begin{minipage}[t]{0.33\linewidth}
    \subsection{Radius}
      \begin{adjustwidth}{2em}{0pt}
      for v $\in$ V \hfill $\mathcal{O} \left( |V| \right)$
      \begin{adjustwidth}{1em}{0pt}
        BFS(v) \hfill $\mathcal{O} \left( |V| + |E| \right)$\\
        e[v] = max(d) \hfill $\mathcal{O} \left( |V| \right)$
        \end{adjustwidth}
        r = min(e) \hfill $\mathcal{O} \left( |V| \right)$
      \end{adjustwidth}
      \ \\
      Overall: \\

      $ \mathcal{O} \left( \left(|V| \times \left( \left( |V| + |E| \right) + |V| \right) \right) + |V| \right)$
  \end{minipage}
\
  \begin{minipage}[t]{0.33\linewidth}
    \subsection{Diameter}
      \begin{adjustwidth}{2em}{0pt}
      for v $\in$ V \hfill $\mathcal{O} \left( |V| \right)$
      \begin{adjustwidth}{1em}{0pt}
        BFS(v) \hfill $\mathcal{O} \left( |V| + |E| \right)$\\
        e[v] = max(d) \hfill $\mathcal{O} \left( |V| \right)$
        \end{adjustwidth}
        d = max(e) \hfill $\mathcal{O} \left( |V| \right)$
      \end{adjustwidth}
      \ \\
      Overall: \\

      $ \mathcal{O} \left( \left(|V| \times \left( \left( |V| + |E| \right) + |V| \right) \right) + |V| \right)$
  %    $ \mathcal{0} \left( |V|^{2} + |V||E| \right) $
  \end{minipage}

\end{document}
